%!TEX TS-program = xelatex
%!TEX encoding = UTF-8 Unicode

\documentclass[12pt]{extarticle}
% extarticle is like article but can handle 8pt, 9pt, 10pt, 11pt, 12pt, 14pt, 17pt, and 20pt text

\def \ititle {Origins of Mind}

\def \isubtitle {Lecture 01}

\def \iauthor {Stephen A. Butterfill}
\def \iemail{s.butterfill@warwick.ac.uk}
\date{}

%for strikethrough
\usepackage[normalem]{ulem}

\input{/Users/arcadia//latex_imports/preamble_steve_handout}

%\bibpunct{}{}{,}{s}{}{,}  %use superscript TICS style bib
%remove hanging indent for TICS style bib
%TODO doesnt work
\setlength{\bibhang}{0em}
%\setlength{\bibsep}{0.5em}


%itemize bullet should be dash
\renewcommand{\labelitemi}{$-$}

\begin{document}

\begin{multicols*}{3}

\setlength\footnotesep{1em}


\bibliographystyle{newapa} %apalike

%\maketitle
%\tableofcontents




%---------------
%--- start paste




Many actions somehow involve multiple agents, as when playing a piano duet or cycling across a park together. Shared agency is that which is exercised in such cases of acting together, whatever it is. How should we give an account of shared agency? Just here things get a little wild in philosophy. Some postulate novel kinds of intentions \citep{Searle:1990em} or modes \citep{gallotti:2013_social}, novel kinds of agents \citep{helm_plural_2008}, and novel kinds of reasoning \citep{Gold:2007zd}. Others suggest embedding intentions in special kinds of commitment \citep{gilbert:2014_book}, or creating nested structures of intention and common knowledge \citep{bratman:2014_book}. Opposing all these ideas, others have proposed understanding shared agency in terms of collective ends \citep{miller_social_2001} or aggregate actions \citep{chant_unintentional_2007}. These accounts of shared agency imply conflicting answers to basic questions such as whether shared agency involves cooperation, non-coercion, commitment, awareness of acting together, or planning capacities. In drawing on philosophers’ ideas to frame hypotheses, scientists have resorted to picking and mixing views, yielding apparently inconsistent background assumptions \citep[e.g.][]{Tomasello:2005wx}. How can we settle which view is correct? As in the first debate about primitive agency so in the cases of shared agency: neither informal observation general metatheoretical principles (e.g. simplicity) seem sufficient to answer this question.



%--- end paste
%---------------

\footnotesize
\bibliography{$HOME/endnote/phd_biblio}

\end{multicols*}

\end{document}
