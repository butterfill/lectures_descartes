%!TEX TS-program = xelatex
%!TEX encoding = UTF-8 Unicode

\documentclass[12pt]{extarticle}
% extarticle is like article but can handle 8pt, 9pt, 10pt, 11pt, 12pt, 14pt, 17pt, and 20pt text

\def \ititle {Descartes}

\def \isubtitle {Lecture 02}

\def \iauthor {Stephen A. Butterfill}
\def \iemail{s.butterfill @warwick.ac.uk}
\date{}

%for strikethrough
\usepackage[normalem]{ulem}

\input{$HOME/latex_imports/preamble_steve_handout}

%\bibpunct{}{}{,}{s}{}{,}  %use superscript TICS style bib
%remove hanging indent for TICS style bib
%TODO doesnt work
\setlength{\bibhang}{0em}
%\setlength{\bibsep}{0.5em}


%itemize bullet should be dash
\renewcommand{\labelitemi}{$-$}

\begin{document}

\begin{multicols*}{3}

\setlength\footnotesep{1em}


\bibliographystyle{newapa} %apalike

%\maketitle
%\tableofcontents




%---------------
%--- start paste
%---------------




      
\def \ititle {Lecture 05}
 
\def \isubtitle {Descartes}
 
\begin{center}
 
{\Large
 
\textbf{\ititle}: \isubtitle
 
}
 
 
 
\iemail %
 
\end{center}


What is it to be led away from the senses?

 
\section{Wax}
 
‘even bodies are not strictly perceived by the senses or the faculty of the imagination but by the
intellect alone’
(Meditation 2)
 
‘I can grasp that the wax is capable of countless changes, yet I am unable to run through this
immeasurable number of changes in my imagination… The nature of this piece of wax is in no way
revealed by my imagination, but is perceived by the mind alone’
(Meditation 2).
 
‘Something which I thought I was seeing with my eyes is in fact grasped solely by the faculty of
judgement which is in my mind’
(Meditation 2).

Argument sketch:
\begin{enumerate}
\item Sensory perceptions of the wax change. 
\item The essential nature of the wax does not.
\item Therefore the senses cannot inform us about its essential nature.
\end{enumerate}

‘The Stoics claimed that each of us has many cognitive impressions,
typically sense impressions of a particular sort, and that these
cognitive impressions are in one way or another the basis for everything
that we can know. A cognitive impression is one that “[1] arises from
what is and [2] is stamped and impressed exactly in accordance with what
is, [3] of such a kind as could not arise from what is not.”’ (Sextus
Empiricus, Against the Logicians 7.248 (1997, 132–33) cited by
\citealp[p.~72]{broughton:2003_descartes}).

 
\section{Error}
 
‘I know by experience that I am prone to countless errors’
\citep[p.~38; AT VII: 54]{descartes:1985_csm2}
 
‘my errors ... are the only evidence of some imperfection in me’
\citep[p.~39; AT VII: 56]{descartes:1985_csm2}
 
‘So what then is the source of my mistakes?
It must be simply this: the scope of the will is wider than that of the intellect; but instead of restricting it within the same limits, I extend its use to matters which I do not understand’ 
\citep[p.~40; AT VII: 58]{descartes:1985_csm2}
 
\emph{The intellect} is the faculty of representation.
\emph{The will} is what affirms or denies somthing represented
\emph{Judgement} occurs when the intellect represents something which the will affirms (or denies).
 
‘If [...] I simply refrain from making a judgement in cases where I do not perceive the truth with sufficient clarity and distinctness, then it is clear that I am behaving correctly and avoiding error.  
But if in such cases I either affirm or deny, then I am not using my free will correctly’
\citep[p.~41; AT VII: 59--60]{descartes:1985_csm2}
 
‘I can avoid error [...] merely [... by] remembering to
withhold judgement on any occasion when the truth of the matter is not clear.’
\citep[p.~43; AT VII: 62]{descartes:1985_csm2}
 
‘today I have learned not only what precautions to take to avoid ever going
wrong,
but also what to do to arrive at the truth.
For I shall unquestionably
reach the truth, if only I give sufficient attention to all the things which I
perfectly understand, and separate these from all the other cases where my
apprehension is more confused and obscure’
\citep[p.~43; AT VII: 62]{descartes:1985_csm2}



\section{Error and The Wax}
The claim that bodies are ‘not strictly perceived by the senses’ plays an essential
role in Descartes account of error. Since the senses do not strictly perceive bodies,
they cannot be the cause of errors about bodies.



% --------------
% ---- end paste
% --------------


% \vfill

% \

% \columnbreak

% \ 
% \vfill

\footnotesize
\bibliography{$HOME/endnote/phd_biblio}

\end{multicols*}

\end{document}
