%!TEX TS-program = xelatex
%!TEX encoding = UTF-8 Unicode

\documentclass[12pt]{extarticle}
% extarticle is like article but can handle 8pt, 9pt, 10pt, 11pt, 12pt, 14pt, 17pt, and 20pt text

\def \ititle {Descartes}

\def \isubtitle {Lecture 02}

\def \iauthor {Stephen A. Butterfill}
\def \iemail{s.butterfill @warwick.ac.uk}
\date{}

%for strikethrough
\usepackage[normalem]{ulem}

\input{$HOME/latex_imports/preamble_steve_handout}

%\bibpunct{}{}{,}{s}{}{,}  %use superscript TICS style bib
%remove hanging indent for TICS style bib
%TODO doesnt work
\setlength{\bibhang}{0em}
%\setlength{\bibsep}{0.5em}


%itemize bullet should be dash
\renewcommand{\labelitemi}{$-$}

\begin{document}

\begin{multicols*}{3}

\setlength\footnotesep{1em}


\bibliographystyle{newapa} %apalike

%\maketitle
%\tableofcontents




%---------------
%--- start paste
%---------------


      
\def \ititle {Lecture 09}
 
\def \isubtitle {Descartes}
 
\begin{center}
 
{\Large
 
\textbf{\ititle}: \isubtitle
 
}
 
 
 
\iemail %
 
\end{center}
 
 
 
\section{Descartes’ Aim in The Meditations}
 
‘these six meditations contain all the foundations of my physics.  But please do not tell people, for that might make it harder for supporters of Aristotle to approve them.  I hope that readers will gradually get used to my principles, and recognize their truth,  before they notice that  they destroy the principles of Aristotle’ 
(Letter to Mersenne, 28 January 1641).
 
 
\section{Aristotelians vs Descartes on the Essential Nature of Bodies}
 
‘The whole of Philosophy is like a tree, of which the  roots are Metaphysics, the trunk is Physics, and the branches which come out of this trunk are all the other sciences ...’ 
\citep[p.~186, AT IX:14]{descartes:1985_csm1}
 
‘The only principles which I accept or require in physics are those of geometry and pure mathematics;
these principles explain all natural phenomena, and enable us to provide quite certain demonstrations 
regarding them’
\citep[p.~247, AT 2:64]{descartes:1985_csm1}
 
External bodies 
‘may not exist in a way that 
exactly corresponds with my  sensory grasp of them, 
for in many cases the 
grasp of the senses
 is very obscure and confused. 
But at least they possess all the properties which  I clearly and distinctly understand, 
that is all those which, viewed in general terms, are comprised within the subject matter of pure mathematics.’
 
‘in the whole history of physics up to now people have
only tried to imagine some causes to explain the phenomena of nature, with
virtually no success.
Compare my assumptions with the assumptions of others.
Compare all their real qualities, their substantial forms, their elements and
countless other such things
with 
my single assumption that 
all bodies are composed of parts’
(\citealp[p.~107]{descartes:1984_vol3} AT 2:199-200 quoted (with wrong page number) by \citealp{sorell:2018_experimental}).
 
 
 
 
 
 
\section{Natural Light vs Natural Impulses}
 
‘whatever is revealed to me by the natural light — for example that from the fact that I
am doubting it follows that I exist, and so on — cannot in any way be open to doubt.
This is because there cannot be another faculty both as trustworthy as the natural
light and also capable of showing me that such things are not true.’
(Third Meditation).
 
‘as for my natural impulses, I have often judged in the past that they were pushing me in the wrong direction when it
was a question of choosing the good, and I do not see why I should place any greater confidence in them in other
matters’
(Third Meditation).
 
‘Dual-process theories of thinking and reasoning quite literally propose the presence of two minds in one brain. The stream of consciousness that broadly corresponds to System 2 thinking is massively supplemented by a whole set of autonomous subsystems in System 1 that post only their final products into consciousness and compete directly for control of our inferences, decisions and actions. However, System 2 provides the basis for hypothetical thinking that endows modern humans with unique potential for a higher level of rationality in their reasoning and decision-making.’
\citep[p.~458]{evans:2003_two}
 
‘the proper purpose of [...] sensory perceptions [...] is simply to inform the mind of what is beneficial or harmful’ 
\citep[pp.~57-8]{descartes:1985_csm2}
 





% --------------
% ---- end paste
% --------------


% \vfill

% \

% \columnbreak

% \ 
% \vfill

\footnotesize
\bibliography{$HOME/endnote/phd_biblio}

\end{multicols*}

\end{document}
