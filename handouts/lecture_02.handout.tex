%!TEX TS-program = xelatex
%!TEX encoding = UTF-8 Unicode

\documentclass[12pt]{extarticle}
% extarticle is like article but can handle 8pt, 9pt, 10pt, 11pt, 12pt, 14pt, 17pt, and 20pt text

\def \ititle {Descartes}

\def \isubtitle {Lecture 02}

\def \iauthor {Stephen A. Butterfill}
\def \iemail{s.butterfill @warwick.ac.uk}
\date{}

%for strikethrough
\usepackage[normalem]{ulem}

\input{$HOME/latex_imports/preamble_steve_handout}

%\bibpunct{}{}{,}{s}{}{,}  %use superscript TICS style bib
%remove hanging indent for TICS style bib
%TODO doesnt work
\setlength{\bibhang}{0em}
%\setlength{\bibsep}{0.5em}


%itemize bullet should be dash
\renewcommand{\labelitemi}{$-$}

\begin{document}

\begin{multicols*}{3}

\setlength\footnotesep{1em}


\bibliographystyle{newapa} %apalike

%\maketitle
%\tableofcontents




%---------------
%--- start paste
%---------------


      
\def \ititle {Lecture 02}
 
\def \isubtitle {Descartes}
 
\begin{center}
 
{\Large
 
\textbf{\ititle}: \isubtitle
 
}
 
 
 
\iemail %
 
\end{center}
 

Sensory perceptions provide only very obscure information about the essential nature of bodies.
How Descartes establish this?

 
\section{The World: Light and Sound}

\subsection{Light}
‘In putting forward an account of light, the first point I want to draw to your attention  is that it is possible for there to be a difference between the sensation that we have of it,  that is, the idea that we form of it in our imagination through the intermediary of our eyes,  and what it is in the objects that produces the sensation in us, that is, what it is in the flame or in the Sun that we term ‘light’’ 
\citep[][p. 81 (AT XI:3)]{descartes:1998_world}
 
\subsection{Signifying Isn’t (Always) Resembling}
‘if words, which signify something only through human convention, are sufficient to make us think of
things to which they bear no resemblance, why could not Nature also have established some sign
which would make us have a sensation of light, even if that sign had in it nothing that resembled
this sensation? And is it not thus that Nature has established laughter and tears, to make us read
joy and sorrow on the face of men?’ 
\citep[][p.~81 (AT XI:4)]{descartes:1998_world}.
 
\subsection{Sound}
‘Do you think that, when we attend solely to the sound of words without attending to their
signification, the idea of that sound which is formed in our thought is at all like the object that
is the cause of it?
A man opens his mouth, moves his tongue, and breathes out:
I see nothing in all
these actions which is in any way similar to the idea of the sound that they cause us to imagine.
And most philosophers maintain that sound is only a certain vibration of the air striking our ears.
Thus if the sense of hearing transmitted to our thought the true image of its object, then instead
of making us think of the sound, it would have to make us think about the motion of the parts of the
air that are vibrating against our ears.’
\citep[][p. 4--5 (AT IX:5)]{descartes:1998_world}
 
 
\subsection{Why Doubt?}
‘I have not brought up these examples to make you believe categorically that the light in the objects
is something different from what it is in our eyes
I merely wanted you to suspect that there might be a difference’
\citep[][p.~82 (AT XI:6)]{descartes:1998_world}.


\section{Doubt}

\subsection{Dreaming}
‘I see plainly that there are never any sure signs by means of which 
being awake can be distinguished from being asleep’ (Meditation I)

‘... the principal reason for doubt, namely my inability to distinguish 
between being asleep and being awake. 
For ... there is a vast difference between the two, in that dreams are 
never linked by memory with all the other actions of life’ (Meditation 6)

‘when I distinctly see where things come from and where and 
when they come to me, and when I can connect my perceptions of them with 
the whole of the rest of my life without a break, 
then I am quite certain that when I encounter these things 
I am not asleep but awake’ (Meditation 6).

Do any considerations about dreaming provide 
‘reasons ... which give us possible grounds for doubt about all things, especially material things, so long as we have no foundation for the sciences other than those we have had up until now’?


\subsection{Cosmic Deception}
‘How do I know that he has not brought it about that there is no earth, 
no sky, no extended thing, no shape, no size, no place, while at the same 
time ensuring that all these things appear to me to exist just as they do now? 
What is more, since I sometimes believe that others go astray in cases 
where they think they have the most perfect knowledge, 
may I not similarly go wrong every time I add two and three or count the 
sides of a square, or in some even simpler matter, if that is imaginable?’

Is this a reason to doubt all things?

\subsection{Why Doubt?}

The usefulness of extensive doubt ‘lies in freeing us from our preconceived opinions,
and providing the easiest route by which the mind may be led away from the senses.’
  
 
\section{Descartes’ Aim in The Meditations}
 
‘these six meditations contain all the foundations of my physics.  But please do not tell people, for that might make it harder for supporters of Aristotle to approve them.  I hope that readers will gradually get used to my principles, and recognize their truth,  before they notice that  they destroy the principles of Aristotle.’ 

 
\section{Aristotlians vs Descartes on the Essential Nature of Bodies}
 
‘in the whole history of physics up to now people have
only tried to imagine some causes to explain the phenomena of nature, with
virtually no success.
Compare my assumptions with the assumptions of others.
Compare all their real qualities, their substantial forms, their elements and
countless other such things
with 
my single assumption that 
all bodies are composed of parts’
(\citealp[p.~107]{descartes:1984_vol3} AT 2:199-200 quoted (with incorrect page number) by \citealp{sorell:2018_experimental}).
 
‘The only principles which I accept or require in physics are those of geometry and pure mathematics;  these principles explain all natural phenomena, and enable us to provide quite certain demonstrations  regarding them’  
\citep[p.~247 AT 2:64]{descartes:1985_csm1}
 
‘The whole of of Philosophy is like a tree, of which the roots are Metaphysics,
the trunk is Physics, and the branches which come out of this trunk are all the
other sciences ...’
\citep[p.~186]{descartes:1985_csm1}
 
Reasons are provided ‘which give us possible grounds for doubt about all things, especially material things,  
so long as we have no foundation for the sciences other than those we have had up until now’
 


    
    

% --------------
% ---- end paste
% --------------


% \vfill

% \

% \columnbreak

% \ 
% \vfill

\footnotesize
\bibliography{$HOME/endnote/phd_biblio}

\end{multicols*}

\end{document}
