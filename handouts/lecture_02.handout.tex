%!TEX TS-program = xelatex
%!TEX encoding = UTF-8 Unicode

\documentclass[12pt]{extarticle}
% extarticle is like article but can handle 8pt, 9pt, 10pt, 11pt, 12pt, 14pt, 17pt, and 20pt text

\def \ititle {Descartes}

\def \isubtitle {Lecture 02}

\def \iauthor {Stephen A. Butterfill}
\def \iemail{s.butterfill @warwick.ac.uk}
\date{}

%for strikethrough
\usepackage[normalem]{ulem}

\input{$HOME/latex_imports/preamble_steve_handout}

%\bibpunct{}{}{,}{s}{}{,}  %use superscript TICS style bib
%remove hanging indent for TICS style bib
%TODO doesnt work
\setlength{\bibhang}{0em}
%\setlength{\bibsep}{0.5em}


%itemize bullet should be dash
\renewcommand{\labelitemi}{$-$}

\begin{document}

\begin{multicols*}{3}

\setlength\footnotesep{1em}


\bibliographystyle{newapa} %apalike

%\maketitle
%\tableofcontents




%---------------
%--- start paste
%---------------


      
\def \ititle {Lecture 02}
 
\def \isubtitle {Descartes}
 
\begin{center}
 
{\Large
 
\textbf{\ititle}: \isubtitle
 
}
 
 
 
\iemail %
 
\end{center}
 





\section{Against Resemblance}
 
Do sensory perceptions resemble their causes?
 
‘In putting forward an account of light, the first thing that I want to draw to your attention 
is that it is possible for there to be a difference between the sensation that we have of it, 
that is, the idea that we form of it in our imagination through the intermediary of our eyes, 
and what it is in the objects that produces the sensation in us, that is, what it is in the flame 
or in the Sun that we term ‘light’.’
\citep[][p. 3 (AT 3)]{descartes:1998_world}
 
‘if words, which signify something only through human convention, are sufficient to make us think of
things to which they bear no resemblance, why could not Nature also have established some sign
which would make us have a sensation of light, even if that sign had in it nothing that resembled
this sensation? And is it not thus that Nature has established laughter and tears, to make us read
joy and sorrow on the face of men?’
\citep[][p. 4 (AT 4)]{descartes:1998_world}
 
‘Do you think that, when we attend solely to the sound of words without attending to their
signification, the idea of that sound which is formed in our thought is at all like the object that
is the cause of it? A man opens his mouth, moves his tongue, and breathes out: I see nothing in all
these actions which is in any way similar to the idea of the sound that they cause us to imagine.
And most philosophers maintain that sound is only a certain vibration of the air striking our ears.4
Thus if the sense of hearing transmitted to our thought the true image of its object, then instead
of making us think of the sound, it would have to make us think about the motion of the parts of the
air that are vibrating against our ears.’
\citep[][p. 4--5 (AT 5)]{descartes:1998_world}
 



    
    

% --------------
% ---- end paste
% --------------


% \vfill

% \

% \columnbreak

% \ 
% \vfill

\footnotesize
\bibliography{$HOME/endnote/phd_biblio}

\end{multicols*}

\end{document}
