%!TEX TS-program = xelatex
%!TEX encoding = UTF-8 Unicode

\documentclass[12pt]{extarticle}
% extarticle is like article but can handle 8pt, 9pt, 10pt, 11pt, 12pt, 14pt, 17pt, and 20pt text

\def \ititle {Descartes}

\def \isubtitle {Lecture 02}

\def \iauthor {Stephen A. Butterfill}
\def \iemail{s.butterfill @warwick.ac.uk}
\date{}

%for strikethrough
\usepackage[normalem]{ulem}

\input{$HOME/latex_imports/preamble_steve_handout}

%\bibpunct{}{}{,}{s}{}{,}  %use superscript TICS style bib
%remove hanging indent for TICS style bib
%TODO doesnt work
\setlength{\bibhang}{0em}
%\setlength{\bibsep}{0.5em}


%itemize bullet should be dash
\renewcommand{\labelitemi}{$-$}

\begin{document}

\begin{multicols*}{3}

\setlength\footnotesep{1em}


\bibliographystyle{newapa} %apalike

%\maketitle
%\tableofcontents




%---------------
%--- start paste
%---------------


      
\def \ititle {Lecture 03}
 
\def \isubtitle {Descartes}
 
\begin{center}
 
{\Large
 
\textbf{\ititle}: \isubtitle
 
}
 
 
 
\iemail %
 
\end{center}
 
Question for this Lecture:
Why, if at all, is doubt necessary to establish ‘anything at all in the sciences that is stable and likely to last’?

\section{Reasons}
Reasons are provided ‘which give us possible grounds for doubt about all things, especially material things,  
so long as we have no foundation for the sciences other than those we have had up until now’

What are these reasons?
 
 
 
\section{Cosmic Doubt}
‘How do I know that he has not brought it about that there is no earth, 
no sky, no extended thing, no shape, no size, no place, while at the same 
time ensuring that all these things appear to me to exist just as they do now? 
What is more, since I sometimes believe that others go astray in cases 
where they think they have the most perfect knowledge, 
may I not similarly go wrong every time I add two and three or count the 
sides of a square, or in some even simpler matter, if that is imaginable?’
 
Is this a reason to doubt all things?

\section{Three Questions}

\begin{enumerate}
\item Is Descartes’ appeal to cosmic deception supposed to provide reasons which give us possible grounds for doubt about all things ?
\item If so, how is Descartes’ appeal to cosmic deception supposed to provide  reasons which give us possible grounds for doubt about all things?
\item Does it succeed?
\end{enumerate}
 
\section{Steve’s attempt to answer Q2}
\begin{enumerate}
\item Sensory perception plus knowledge of platitudes alone do not enable you to know that you aren’t cosmically deceived.
\item You do know this platitude: if you are drinking coffee, then you are not cosmically deceived
\item Suppose (for a contradiction) that sensory perception alone enables you to know you are drinking coffee.
\item Then you would be in a position to know you are not cosmically deceived on the basis of sensory perception plus knowledge of platitudes only.
\item Therefore sensory perception alonedoes not enable you to know you are drinking coffee.
\end{enumerate}
 
\section{Why doubt?}
‘I had seen many ancient writings by the Academics and Sceptics on this subject, 
and was reluctant to reheat and serve this precooked material’
\citep[p.~94, AT VII:130]{descartes:1985_csm2}
 
The usefulness of extensive doubt ‘lies in freeing us from our preconceived opinions, 
and providing the easiest route by which the mind may be led away from the senses.’
 

    



% --------------
% ---- end paste
% --------------


% \vfill

% \

% \columnbreak

% \ 
% \vfill

\footnotesize
\bibliography{$HOME/endnote/phd_biblio}

\end{multicols*}

\end{document}
