%!TEX TS-program = xelatex
%!TEX encoding = UTF-8 Unicode

\documentclass[12pt]{extarticle}
% extarticle is like article but can handle 8pt, 9pt, 10pt, 11pt, 12pt, 14pt, 17pt, and 20pt text

\def \ititle {Descartes}

\def \isubtitle {Lecture 02}

\def \iauthor {Stephen A. Butterfill}
\def \iemail{s.butterfill @warwick.ac.uk}
\date{}

%for strikethrough
\usepackage[normalem]{ulem}

\input{$HOME/latex_imports/preamble_steve_handout}

%\bibpunct{}{}{,}{s}{}{,}  %use superscript TICS style bib
%remove hanging indent for TICS style bib
%TODO doesnt work
\setlength{\bibhang}{0em}
%\setlength{\bibsep}{0.5em}


%itemize bullet should be dash
\renewcommand{\labelitemi}{$-$}

\begin{document}

\begin{multicols*}{3}

\setlength\footnotesep{1em}


\bibliographystyle{newapa} %apalike

%\maketitle
%\tableofcontents




%---------------
%--- start paste
%---------------





      

    
      
\def \ititle {Lecture 04}
 
\def \isubtitle {Descartes}
 
\begin{center}
 
{\Large
 
\textbf{\ititle}: \isubtitle
 
}
 
 
 
\iemail %
 
\end{center}
 
‘The main reason why we can find nothing in ordinary philosophy which is so evident and certain as to be beyond dispute is that students of the subject first of all are not content to acknowledge what is clear and certain, but on the basis of merely probably conjectures venture also to make assertions on obscure matters about which nothing is known; they then gradually come to have complete faith in these assertions, ... The result is that the only conclusions they can draw are ones which apparently rest on some such obscure proposition, and which are accordingly uncertain.’ (Rules for the Direction of the Mind, p. 14)
\citep[p.~14, AT X:367--8]{descartes:1985_csm1}
 
‘in practical life 
it is sometimes necessary to act upon opinons which one knows to be quite uncertain just as if they were indubitable’
\citep[p.~126 AT 6:31]{descartes:1985_csm1}
 
In devoting ‘myself soley to the search for truth
... I resolved to 
pretend
 that all the things that had ever
entered my mind were no more true
that the illusions of my dreams’
\citep[p.~127 AT 6:32]{descartes:1985_csm1}
 
‘I noticed that while I was trying thus to think everything false
it was necessary that I,
who was thinking this,
was something.’
\citep[p.~126 AT 6:31]{descartes:1985_csm1}
 
‘this truth
“I am thinking, therefore I exist”
[is] so firm and sure 
that all the most extravagant suppositions of the sceptics [are] incapable of shaking it’
 
I took it as ‘the first principle’.
\citep[p.~126 AT 6:31]{descartes:1985_csm1}
 
‘We clearly understand that it is possible for me to exist at this moment, while I am thinking of
one thing, and yet not to exist at the very next moment’
\citep[p.~355 AT V:192]{descartes:1984_vol3}
 
‘When someone says 'I am breathing, therefore I exist', if he wants to prove he exists from the fact
that there cannot be breathing without existence, he proves nothing, because he would have to prove
first that it is true that he is breathing, which is impossible unless he has also proved that 
he exists’ 
\citep[p.~98 AT II:37]{descartes:1984_vol3}
 
‘When someone says “I am thinking, therefore I am, or I exist,” he does not deduce existence from thought by means of a syllogism, but by a simple intuition of the mind.  
This is clear from the fact that if he were deducing it by a syllogism, he would
previously have had to know the major premise “Everything that thinks is, or exists”
yet in fact
he learns this from experiencing in his own case that it is impossible that he should think
without existing.’
\citep[AT 7:140]{descartes:1985_csm2}
 
‘...  the truth of the proposition 'I am thinking, therefore I exist.' 
Now this knowledge is not the work
of your reasoning
[...] it is something that your mind sees,
feels and handles; 
[...] although your imagination insistently mixes itself up with your thoughts
and lessens the clarity of this knowledge by trying to clothe it with shapes
\citep[p.~331 AT V:138]{descartes:1984_vol3}
 

    

% --------------
% ---- end paste
% --------------


% \vfill

% \

% \columnbreak

% \ 
% \vfill

\footnotesize
\bibliography{$HOME/endnote/phd_biblio}

\end{multicols*}

\end{document}
