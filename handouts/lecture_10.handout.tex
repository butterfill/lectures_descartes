%!TEX TS-program = xelatex
%!TEX encoding = UTF-8 Unicode

\documentclass[12pt]{extarticle}
% extarticle is like article but can handle 8pt, 9pt, 10pt, 11pt, 12pt, 14pt, 17pt, and 20pt text

\def \ititle {Descartes}

\def \isubtitle {Lecture 02}

\def \iauthor {Stephen A. Butterfill}
\def \iemail{s.butterfill @warwick.ac.uk}
\date{}

%for strikethrough
\usepackage[normalem]{ulem}

\input{$HOME/latex_imports/preamble_steve_handout}

%\bibpunct{}{}{,}{s}{}{,}  %use superscript TICS style bib
%remove hanging indent for TICS style bib
%TODO doesnt work
\setlength{\bibhang}{0em}
%\setlength{\bibsep}{0.5em}


%itemize bullet should be dash
\renewcommand{\labelitemi}{$-$}

\begin{document}

\begin{multicols*}{3}

\setlength\footnotesep{1em}


\bibliographystyle{newapa} %apalike

%\maketitle
%\tableofcontents




%---------------
%--- start paste
%---------------




      
\def \ititle {Lecture 10}
 
\def \isubtitle {Descartes}
 
\begin{center}
 
{\Large
 
\textbf{\ititle}: \isubtitle
 
}
 
 
 
\iemail %
 
\end{center}
 
 
 
\section{Is there a Cartesian Circle?}

\subsection{God therefore knowledge}
‘[F]rom this contemplation of the true God [\dots] I think I can see a way forward to the knowledge
of other things. To begin with, I recognize that it is impossible that God
should ever deceive me. [\dots] And since God does not wish to deceive me, he
surely did not give me the kind of faculty which would ever enable me to go
wrong while using it correctly’
\citep[p.~37, AT VII:53–54]{descartes:1985_csm2}
 
‘what I took just now as a rule, namely that everything we
conceive very clearly and very distinctly is true, is assured only for the
reasons that God is or exists, that he is a perfect being, and that
everything in us comes from him. It follows that our ideas or notions,
being real things and coming from God, cannot be anything but true, in
every respect in which they are clear and distinct.’
\citep[p.~130, AT VI:38]{descartes:1985_csm1}
 
\subsection{Knowledge therefore God}
And from the mere fact that there is such an
idea within me [...] I clearly infer that God
also exists [...] 
So clear is this conclusion that I am confident that the human intellect
cannot know anything that is more evident or more certain.’
 
\subsection{Steve’s rough resolultion}
\begin{enumerate}
\item If we can know anything, then we can know God exists.
\item We can know something.
\item Therefore, we can know God exists.
\item But God doesn’t deceive us.
\item Therefore what we clearly perceive is true.
\end{enumerate}
Compare \citet{murdoch:1999_cartesian}: ‘not only is the doubt-insinuating thought not a reason 
for Descartes to doubt his entitlement to infer that God exists, but also he has no other 
reason to doubt this.’ Conflicting interpretations include \citet{rocca:2005_descartes,broughton:2003_descartes,doney:1955_cartesian}.

\subsection{Evidence against Steve’s resolution}
‘an atheist can be clearly aware that the three angles of a triangle are equal to two 
right angles [...]. But I maintain that this awareness of his is not 
true knowledge, since no act of awareness that can be rendered doubtful seems 
fit to be called knowledge’
\citep[p.~101, AT VII: 141]{descartes:1985_csm2}
 
‘as to the fact that I was not guilty of circularity when I said that the only
reason we have for being sure that what we clearly and distinctly perceive is
true is the fact that God exists, but that we are sure that God exists only
because we perceive this clearly: [...] I made a distinction between what we in fact perceive
clearly and what we remember having perceived clearly on a previous occasion.
To begin with, we are sure that God exists because we attend to the arguments
which prove this; but subsequently it is enough for us to remember that we
perceived something clearly in order for us to be certain that it is true. This
would not be sufficient if we did not know that God exists and is not a
deceiver’
\citep[p.~171, AT VII:245--6]{descartes:1985_csm2}
 
Both passages support an interpretation like that in \citet{williams:2014_descartes}.
 
\section{A Puzzle about the Senses}

\emph{Dilemma} On Descartes’ view, do the senses represent things?
If so, how is it that the senses never misrepresent things? (Or, if they do sometimes misrepresent, why are they not a source of error)?
If not, why must we ‘not judge that external things always are just as they appear to be’?

Sensory perceptions of tastes,
smells, sounds, heat, cold, light, colors and the like ‘do not represent
anything located outside our thought’
(\citealp[p.~ 219, AT VIII:35]{descartes:1985_csm1} cited by \citealp[p.~348]{simmons:1999_are})
  
Sensory perceptions ‘normally tell us of the benefit or harm that external bodies may do [\dots], and do not, except [\dots] accidentally, show us what external bodies are like in themselves’ 
(\citealp[p.~224, AT VIII: 41]{descartes:1985_csm1} cited by \citealp[p.~350]{simmons:1999_are}).
 
External bodies 
‘may not exist in a way that 
exactly corresponds with my  sensory grasp of them’
 
‘the intellect can never be deceived by any experience [\dots].
Furthermore, it must not judge that the imagination faithfully represents the objects of
the senses, or that the senses take on the true shapes of things, or in short that
external things always are just as they appear to be. 
In all such cases we are liable to go wrong, as we do for example when we take as gospel
truth a story which someone has told us; or as someone who has jaundice does when, owing
to the yellow tinge of his eyes, he thinks everything is coloured yellow; or again, as we
do when our imagination is impaired (as it is in depression) and we think that its
disordered images represent real things. 
But the understanding of the wise man will not be deceived in such cases: while he will
judge that whatever comes to him from his imagination really is depicted
in it, he will never assert that it passes, complete and unaltered, from the external
world to his senses, and from his senses to the corporeal imagination’
(Rules for the Direction of the Mind, Rule 12)
\citep[p.~47, AT X:423]{descartes:1985_csm1}.
 
‘consider the reasons why [vision]
sometimes deceives us. First, it is the soul which sees, and not the eye;
and it does not see directly, but only by means of the brain. That is why
madmen and those who are asleep often see, or think they see, various
objects which are nevertheless not before their eyes: namely, certain
vapours disturb their brain and arrange those of its parts normally
engaged in vision exactly as they would be if these objects were present.
Then, because the impressions which come from outside pass to the
'common' sense by way of the nerves, if the position of these nerves is
changed by any unusual cause, this may make us see objects in places
other than where they are [\dots] Again, because we normally judge that the 
impressions which stimulate our sight come from places towards which
we have to look in order to sense them, we may easily be deceived when
they happen to come from elsewhere. Thus, those whose eyes are affected
by jaundice, or who are looking through yellow glass or shut up in a
room where no light enters except through such glass, attribute this
colour to all the bodies they look at. And the person inside the dark room
which I described earlier attributes to the white body the colours of the
objects outside because he directs his sight solely upon that body. And if
our eyes see objects through lenses and in mirrors, they judge them to be
at points where they are not and to be smaller or larger than they are, or
inverted as well as smaller (namely, when they are somewhat distant
from the eyes). This occurs because the lenses and mirrors deflect the rays’
\emph{Optics}
\citep[pp.~172--3, AT VI:141--2]{descartes:1985_csm1}.
 
 
 
\section{Conclusion}
 
‘although I feel heat when I go near a fire and feel pain when I go too near,
there is no convincing argument for supposing that there is something in the
fire which resembles the heat, any more than for supposing that there is
something which resembles the pain.
There is simply reason to suppose that
there is something in the fire, whatever it may eventually turn out to be,
which produces in us the feelings of heat or pain’
\citep[p.~58, AT VII:83]{descartes:1985_csm2}
 
‘... the principal reason for doubt, namely my inability
to distinguish between being asleep and being awake. For I now notice that
there is a vast difference between the two, in that dreams are never linked by
memory with all the other actions of life as waking experiences are.
If, while
I am awake, anyone were suddenly to appear to me and then disappear
immediately, as happens in sleep, so that I could not see where he had come
from or where he had gone to, it would not be unreasonable for me to judge that
he was a ghost, or a vision created in my brain, rather than a real man. But
when I distinctly see where things come from and where and when they come to
me, and when I can connect my perceptions of them with the whole of the rest of
my life without a break, then I am quite certain that when I encounter these
things I am not asleep but awake. And I ought not to have even the slightest
doubt of their reality if, after calling upon all the senses as well as my
memory and my intellect in order to check them, I receive no conflicting
reports from any of these sources. For from the fact that God is not a deceiver
it follows that in cases like these I am completely free from error.’
\citep[p.~61--2, AT VII:89--90]{descartes:1985_csm2}
 
‘I wanted to show the firmness of the truths which I propound later on, in the
light of the fact that they cannot be shaken by these metaphysical doubts.
...
I could not have left them out, any more than a medical writer can leave out the description of a disease when he wants to explain how it can be cured.’ 
\citep[p.~121, AT VII:172]{descartes:1985_csm2}
 

    

% --------------
% ---- end paste
% --------------


% \vfill

% \

% \columnbreak

% \ 
% \vfill

\footnotesize
\bibliography{$HOME/endnote/phd_biblio}

\end{multicols*}

\end{document}
