%!TEX TS-program = xelatex
%!TEX encoding = UTF-8 Unicode

\documentclass[12pt]{extarticle}
% extarticle is like article but can handle 8pt, 9pt, 10pt, 11pt, 12pt, 14pt, 17pt, and 20pt text

\def \ititle {Descartes}

\def \isubtitle {Lecture 02}

\def \iauthor {Stephen A. Butterfill}
\def \iemail{s.butterfill @warwick.ac.uk}
\date{}

%for strikethrough
\usepackage[normalem]{ulem}

\input{$HOME/latex_imports/preamble_steve_handout}

%\bibpunct{}{}{,}{s}{}{,}  %use superscript TICS style bib
%remove hanging indent for TICS style bib
%TODO doesnt work
\setlength{\bibhang}{0em}
%\setlength{\bibsep}{0.5em}


%itemize bullet should be dash
\renewcommand{\labelitemi}{$-$}

\begin{document}

\begin{multicols*}{3}

\setlength\footnotesep{1em}


\bibliographystyle{newapa} %apalike

%\maketitle
%\tableofcontents




%---------------
%--- start paste
%---------------


      
\def \ititle {Lecture 07}
 
\def \isubtitle {Descartes}
 
\begin{center}
 
{\Large
 
\textbf{\ititle}: \isubtitle
 
}
 
 
 
\iemail %
 
\end{center}
 
 
 
\section{The Senses and Appearances}
 
Sensory perceptions of tastes, smells, sounds, heat, cold, light, colors and the like ‘do not represent anything located outside our thought’  [i.e. lack intentional objects]
These sensory perceptions ‘vary according to the different movements which pass 
from all parts of our body to the ... brain’
(\citealp[p.~ 219, AT VIII:35]{descartes:1985_csm1} cited by \citealp[p.~348]{simmons:1999_are})
 
‘Something which I thought I was seeing with my eyes is in fact grasped solely by the faculty of
judgement which is in my mind’
(Meditation 2).


\subsection{Error and Perceptual Appearances}
\begin{enumerate}
\item If there are perceptual appearances, then sensory perceptions can misrepresent things.
\item If sensory perceptions can misrepresent things, then they are a source of error.
\item Descartes holds that the will is the sole source of error.
\item Therefore Descartes must deny that there are perceptual appearances.
\end{enumerate}

\subsection{Descartes’ Three Grades of Sensory Response}
‘when I see a stick, it should not be supposed that certain ‘intentional
forms’ fly off the stick towards the eye, but simply that rays of light
are reflected off the stick and set up certain movements in the optic nerve
and, via the optic nerve, in the brain, as I have explained at some length
in the Optics. This movement in the brain, which is common to us and the
brutes, is the \textbf{first grade of sensory response}. This leads to 
\textbf{the second
grade}, which extends to the mere perception of the colour and light
reflected from the stick; it arises from the fact that the mind is so
intimately conjoined with the body that it is affected by the movements
which occur in it. 
\emph{Nothing more than this should be referred to the sensory
faculty, if we wish to distinguish it carefully from the intellect.} 
But
suppose that, as a result of being affected by this sensation of colour, I
judge that a stick, located outside me, is coloured; and suppose that on
the basis of the extension of the colour and its boundaries together with
its position in relation to the parts of the brain, I make a rational
calculation about the size, shape and distance of the stick: although such
reasoning is commonly assigned to the senses (which is why I have here
referred it to \textbf{the third grade of sensory response}), it is clear that it
depends solely on the intellect’
\citep[p.~295, AT VII:437]{descartes:1985_csm2}.
 

\subsection{Questions Arising}

\begin{enumerate}
\item Does reflection on appearances show that sensory perceptions represent?
\item Or do scientific discoveries show that sensory perceptions represent?
\item If so, is Descartes wrong about error?
\end{enumerate}



                

                

              
 
 
\section{Clear and Distinct}

‘The only principles which I accept or require in physics are those of geometry and pure mathematics; 
these principles explain all natural phenomena, and enable us to provide quite certain demonstrations 
regarding them’
\citep[p.~247, AT 2:64]{descartes:1985_csm1}
 
‘I now seem to be able to lay it down as a general rule that
whatever I perceive clearly and distinctly is true’
(Meditation 3)
 
‘the perception I have of it is a case not of 
vision or touch or imagination 
... but of 
purely mental scrutiny;
 
‘\emph{What is meant by a clear perception, and by a distinct perception.}
I
call a perception 'clear' when it is present and accessible to the
attentive mind - just as we say that we see something clearly when it is
present to the eye's gaze and stimulates it with a sufficient degree of
strength and accessibility.
I call a perception 'distinct' if, as well as
being clear, it is so sharply separated from all other perceptions that it contains
within itself only what is clear’
\citep[pp.~207--8, AT VIII:21--22]{descartes:1985_csm1}
 
‘The [...] natural light [...] enables me to perceive that I would have given myself all the perfections of which I have an idea, if I had given myself existence’
(Fourth Replies).
 
‘the learned
often employ distinctions so subtle that they disperse the natural light,
and they detect obscurities even in matters which are perfectly clear to
peasants’
\citep[p.~59]{descartes:1985_csm1}
 

    



% --------------
% ---- end paste
% --------------


% \vfill

% \

% \columnbreak

% \ 
% \vfill

\footnotesize
\bibliography{$HOME/endnote/phd_biblio}

\end{multicols*}

\end{document}
