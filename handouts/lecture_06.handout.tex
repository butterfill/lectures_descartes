%!TEX TS-program = xelatex
%!TEX encoding = UTF-8 Unicode

\documentclass[12pt]{extarticle}
% extarticle is like article but can handle 8pt, 9pt, 10pt, 11pt, 12pt, 14pt, 17pt, and 20pt text

\def \ititle {Descartes}

\def \isubtitle {Lecture 02}

\def \iauthor {Stephen A. Butterfill}
\def \iemail{s.butterfill @warwick.ac.uk}
\date{}

%for strikethrough
\usepackage[normalem]{ulem}

\input{$HOME/latex_imports/preamble_steve_handout}

%\bibpunct{}{}{,}{s}{}{,}  %use superscript TICS style bib
%remove hanging indent for TICS style bib
%TODO doesnt work
\setlength{\bibhang}{0em}
%\setlength{\bibsep}{0.5em}


%itemize bullet should be dash
\renewcommand{\labelitemi}{$-$}

\begin{document}

\begin{multicols*}{3}

\setlength\footnotesep{1em}


\bibliographystyle{newapa} %apalike

%\maketitle
%\tableofcontents




%---------------
%--- start paste
%---------------



      
\def \ititle {Lecture 06}
 
\def \isubtitle {Descartes}
 
\begin{center}
 
{\Large
 
\textbf{\ititle}: \isubtitle
 
}
 
 
 
\iemail %
 
\end{center}
 
 
 
\section{Error and the Senses}
 
Sensory perceptions of tastes,
smells, sounds, heat, cold, light, colors and the like ‘do not represent
anything located outside our thought’
These sensory perceptions ‘vary according to the different movements which pass 
from all parts of our body to the ... brain’
(\citealp[p.~ 219, AT VIII:35]{descartes:1985_csm1} cited by \citealp[p.~348]{simmons:1999_are})
 
‘Something which I thought I was seeing with my eyes is in fact grasped solely by the faculty of
judgement which is in my mind’
(Meditation 2).
 
‘[T]he proper purpose of [...] sensory perceptions [...] is simply to inform the mind of what is beneficial or harmful [...]’ 
\citep[pp.~57-8]{descartes:1985_csm2}
 
Distinguish two claims:
\begin{enumerate}
    \item Sensory perceptions are caused by things. 
    \item Sensory perceptions represent (or present) things.
\end{enumerate}
Descartes accepts the first of these two claims (‘I do not see how God could be understood to be anything but a deceiver if the ideas were transmitted from a source other than corporeal things’).
 
Sensory perceptions ‘normally tell us of the benefit or harm that external bodies may do [...], and do not, except occasionally and accidentally, show us what external bodies are like in themselves’ 
(\citealp[p.~224, AT VIII: 41]{descartes:1985_csm1} cited by \citealp[p.~350]{simmons:1999_are}).
 
External bodies 
‘may not exist in a way that 
exactly corresponds with my  sensory grasp of them, 
for in many cases the 
grasp of the senses
 is very obscure and confused. 
But at least they possess all the properties which I clearly and distinctly understand, 
that is all those which, viewed in general terms, are comprised within the subject matter of pure mathematics.’
 
\subsection{Descartes’ Three Grades of Sensory Response}
‘when I see a stick, it should not be supposed that certain ‘intentional
forms’ fly off the stick towards the eye, but simply that rays of light
are reflected off the stick and set up certain movements in the optic nerve
and, via the optic nerve, in the brain, as I have explained at some length
in the Optics.' This movement in the brain, which is common to us and the
brutes, is the \textbf{first grade of sensory response}. This leads to 
\textbf{the second
grade}, which extends to the mere perception of the colour and light
reflected from the stick; it arises from the fact that the mind is so
intimately conjoined with the body that it is affected by the movements
which occur in it. 
\emph{Nothing more than this should be referred to the sensory
faculty, if we wish to distinguish it carefully from the intellect.} 
But
suppose that, as a result of being affected by this sensation of colour, I
judge that a stick, located outside me, is coloured; and suppose that on
the basis of the extension of the colour and its boundaries together with
its position in relation to the parts of the brain, I make a rational
calculation about the size, shape and distance of the stick: although such
reasoning is commonly assigned to the senses (which is why I have here
referred it to \textbf{the third grade of sensory response}), it is clear that it
depends solely on the intellect’
\citep[p.~295, AT VII:437]{descartes:1985_csm2}.

 
\section{How to Write an Essay}
 
‘these six meditations contain all the foundations of my physics.  But please do not tell people’ (Letter to Mersenne)
 
 
\section{Clear and Distinct}
 
‘I now seem to be able to lay it down as a general rule that
whatever I perceive clearly and distinctly is true’
(Meditation 3)
 
‘the perception I have of it is a case not of 
vision or touch or imagination 
... but of 
purely mental scrutiny;
 
‘whatever is revealed to me by the natural light — for example that from the fact that I
am doubting it follows that I exist, and so on — cannot in any way be open to doubt.
This is because there cannot be another faculty both as trustworthy as the natural
light and also capable of showing me that such things are not true.’
(Third Meditation).
 
‘as for my natural impulses, I have often judged in the past that they were pushing me in the wrong direction when it
was a question of choosing the good, and I do not see why I should place any greater confidence in them in other
matters’
(Third Meditation).
 
‘What is meant by a clear perception, and by a distinct perception.
I
call a perception 'clear' when it is present and accessible to the
attentive mind - just as we say that we see something clearly when it is
present to the eye's gaze and stimulates it with a sufficient degree of
strength and accessibility.
I call a perception 'distinct' if, as well as
being clear, it is so sharply separated from all other perceptions that it contains
within itself only what is clear’
\citep[pp.~207--8, AT VIII:21--22]{descartes:1985_csm1}
 
    


% --------------
% ---- end paste
% --------------


% \vfill

% \

% \columnbreak

% \ 
% \vfill

\footnotesize
\bibliography{$HOME/endnote/phd_biblio}

\end{multicols*}

\end{document}
